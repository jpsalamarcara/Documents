% !TEX encoding = UTF-8 Unicode
% This is based on the LLNCS.DEM the demonstration file of
% the LaTeX macro package from Springer-Verlag
% for Lecture Notes in Computer Science,
% version 2.4 for LaTeX2e as of 16. April 2010
%
% See http://www.springer.com/computer/lncs/lncs+authors?SGWID=0-40209-0-0-0
% for the full guidelines.
%

\documentclass{llncs}
\usepackage[utf8]{inputenc}


\begin{document}

\title{Programación Genética Funcional Inductiva en Funico}
%
\titlerunning{Programación Genética}  % abbreviated title (for running head)
%                                     also used for the TOC unless
%                                     \toctitle is used
%
\author{Juan Pablo Salamanca Ramírez}
%
\authorrunning{Juan P. Salamanca} % abbreviated author list (for running head)
%
%%%% list of authors for the TOC (use if author list has to be modified)
\tocauthor{}
%
\institute{Departamento de Ingenieria de Sistemas e Industrial, Universidad Nacional de Colombia, Bogotá, Carrera 45 No 26-85, Colombia,\\
\email{jpsalamarcara@unal.edu.co}}

\maketitle

\begin{abstract}
Genetic programming is a research field of evolutionary computing, in which, by using genetic algorithms, is intended to optimize a population of individuals belonging to the domain of any programming language, this is done according to a certain fitness function.
In this paper, an experimental study of the different evolutionary strategies that can be adopted to apply genetic programming in the domain of Funico functional programming language occurs.
\keywords{evolutionary computation, genetic programing, inductive, functional}
\end{abstract}
%



\section{Introducción}
La programación genética es un campo de investigación de la computación evolutiva, en el cual, mediante el uso de algoritmos genéticos, se pretende optimizar una población de individuos pertenecientes al dominio de algún lenguaje de programación, esto se hace de acuerdo a una determinada función de aptitud.
En este documento, se presenta un estudio experimental acerca de las diferentes estrategias evolutivas que se pueden adoptar para aplicar la programación genética particularmente en el dominio del lenguaje de programación funcional Funico \cite{cub:gom}, desarrollado por Edwin C. Cubides.

%Fin de la Introducción
\section{Estado del Arte}
Desde las primeras propuestas hechas por Koza \cite{koza}, se han presentado continuos avances \cite{koza:1} en materia de programación genética. En el caso de la programación genética inductiva, recientemente han surgido aplicaciones que tratan problemas específicos, como por ejemplo, la detección de reglas de asociación difusas \cite{gonz} y el diseño de motores \cite{karim}.
Por otra parte, con la introducción del algoritmo HaEA \cite{gomez} en la programación genética inductiva \cite{cub:gom:2}, se permitió un gran desarrollo debido a su gran capacidad de optimización y carencia de parámetros para ajustar. Los resultados pronto serán publicados y se espera que a partir de; se descubran nuevas técnicas y mejoras en los distintos campos de aplicación.

\section{Propuesta}
Teniendo en cuenta las principales conclusiones de J.Gómez \cite{gomez} en relación a la preservación de la diversidad como estrategia evolutiva y al análisis de los diferentes esquemas de selección realizado por T. Blicke \& L. Thiele\cite{blick:thiele}; se plantea encontrar un método para seleccionar individuos de la población, en el cual se favorezca la   exploración y explotación, como principios básicos para disminuir la perdida de diversidad y la probabilidad de estancamiento en un mínimo/máximo local. Es así, como se propone un híbrido entre ranking y ruleta.

\subsection{Estrategia de Selección: Ruleta de Rango Estocástico}



\subsection{Estrategia de Reemplazo}

\section{Experimentos y Resultados}


\section{Conclusiones}

\begin{thebibliography}{8}
%
\bibitem {cub:gom}
Edwin, C., Cubides, G. \& Jonatan Gómez P.:
Curso de Computación Evolutiva. Programación Funcional con el Lenguaje Funico.
Bogotá (2015)

\bibitem {koza}
Jhon R., Koza.:
Dynamic, Genetic and Chaotic Programming,
Jhon Wiley and Sons, Inc (1992)

\bibitem {koza:1}
Jhon R., Koza. et al:
Advances in Genetic Programming, 
The MIT Press, ISBN0-262-11188-8 (1994)

\bibitem {gonz}
Antonio, G. et al:
An efficient Inductive Genetic Learning Algorithm for Fuzzy Relational Rules.
International Journal of Computational Intelligence Systems, (2012)

\bibitem {karim}
S, Karim Tanatabaei et al:
Self-adjusting multidisciplinary design of hydraulic engine mount using bond graphs and inductive genetic programming.
Science Direct, (2016)

\bibitem {gomez}
J. Gómez:
Self Adaptation of Operator Rates in Evolutionary Algorithms
Universidad Nacional de Colombia and The University of Memphis, (2004)

\bibitem {cub:gom:2}
Edwin, C., Cubides, G. \& Jonatan Gómez P.:
Curso de Computación Evolutiva. Programación Genética Funcional Inductiva con el Lenguaje Funico.
Bogotá (2015)

\bibitem {blick:thiele}
Tobias Blickle \& Lothar Thiele:
A Comparison of Selection Schemes used in Genetic Algorithms.
Computer Engineering and Communication Networks Lab.
Swiss Federal Institute of Technology. (1995)
Bogotá (2015)

\end{thebibliography}

\end{document}

